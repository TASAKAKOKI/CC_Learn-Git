git remote -v
    Nice work! We have a clone of Sally’s remote on our computer. One thing that Git does behind the scenes when you clone science-quizzes is give the <<remote address>> the name <<origin>>, so that you can refer to it more conveniently. In this case, Sally’s remote is origin.

    You can see a list of a Git project’s remotes with the command:
        git remote -v

Instructions
    1.
    Using the file navigator, examine the contents of the cloned Git project. There are a few quiz files here, which we will be working with during this lesson.

    Open a file of your choice in the code editor.

    2.
    Change directories into the my-quizzes directory, enter this command on the terminal:
        cd my-quizzes
    
    To learn more about cd, take a look at our command line course.

    3.
    Enter git remote -v to list the remotes.

    Notice the output:
        origin    /home/ccuser/workspace/curriculum/science-quizzes (fetch)
        origin    /home/ccuser/workspace/curriculum/science-quizzes (push)
    
        - Git lists the "name of the remote, origin", as well as "its location".
        
        - Git automatically names this remote origin, because it refers to the remote repository of origin. However, it is possible to safely change its name.
    
        - The remote is listed twice: once for ("fetch") and once for ("push"). We’ll learn about these later in the lesson.