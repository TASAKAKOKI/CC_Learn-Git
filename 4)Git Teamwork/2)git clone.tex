git clone
    Sally has created the remote repository, science-quizzes in the directory curriculum, which teachers on the school’s shared network have access to. In order to get your own replica of science-quizzes, you’ll need to clone it with:
        git clone remote_location clone_name

    In this command:
        <<remote_location>> tells Git where to go to find the remote. This could be a web address, or a filepath, such as:
    
        /Users/teachers/Documents/some-remote
    
        <<clone_name>> is the name you give to the directory in which Git will clone the repository.

Instructions
    1.
    The Git remote Sally started is called:
        science-quizzes
    
    Enter the command to clone this remote. Name your clone:
        my-quizzes
    
    Notice the output:
        cloning into 'my-quizzes'...
    
    Git informs us that it’s copying everything from science-quizzes into the my-quizzes directory.

    my-quizzes is your local copy of the science-quizzes Git project. If you commit changes to the project here, Sally will not know about them.