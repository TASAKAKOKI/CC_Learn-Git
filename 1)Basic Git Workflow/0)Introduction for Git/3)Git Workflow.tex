Git Workflow
    Nice! We have a Git project. A Git project can be thought of as having three parts:
        1) "A Working Directory": where you’ll be doing all the work: creating, editing, deleting and organizing files
            git init
        2) "A Staging Area": where you’ll list changes you make to the working directory
            git add
        3) "A Repository": where Git "permanently stores those changes" as different versions of the project
            git commit
            
    The Git workflow consists of "editing files in the working directory", "adding files to the staging area", and "saving changes to a Git repository". In Git, we save changes with a commit, which we will learn more about in this lesson.

Instructions
    Take a look at the diagram. Before we move on, it will help to be familiar with the three parts of the Git workflow. Click Next to continue.