delete branch
    In Git, branches are usually a means to an end. You create them to work on a new project feature, but the end goal is to merge that feature into the master branch. After the branch has been integrated into master, it has served its purpose and can be deleted.

    The command
        git branch -d branch_name
        
    will delete the specified branch from your Git project.

    Now that master contains all the file changes that were in fencing, let’s delete fencing.

Instructions
    1.  
    Delete the fencing branch.

    Now, verify that you have indeed deleted fencing by listing all your project’s branches on the terminal.

    Notice in the output that only one branch, master, is shown.

    Click Next to continue!